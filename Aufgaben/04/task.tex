\task{Grenzwerte berechnen ohne die Regeln von l'Hospital}{2+2+2+2}
Berechne (ohne Anwendung der de l'Hospital Regel) folgende Grenzwerte, falls diese existieren.
\begin{minipage}[t]{0.5\textwidth}
    \begin{subtask}
        $\lim_{x\to0}\frac{1-cos(x)}{sin^2(x)}$
    \end{subtask}
    \begin{solution}
        \[
            \lim_{x\to0}\frac{1-cos(x)}{sin^2(x)}
            =\lim_{x\to0}\frac{1-\sum_{k=0}^{\infty}(-1)^n\frac{x^{2k}}{(2k)!}}{\left(\sum_{k=0}^{\infty}(-1)^n\frac{x^{2k+1}}{(2k+1)!}\right)^2}
        \]
        \[
            =\lim_{x\to0}\frac{1-(1-\frac{x^2}{2}+\frac{x^4}{24}-...)}{(x-\frac{x^3}{6}+...)^2}
        \]
        \[
            =\lim_{x\to0}\frac{\frac{x^2}{2}-\frac{x^4}{24}+...}{x^2-2(\frac{x^4}{6}+\frac{x^6}{12}-...)+(\frac{x^4}{6}+\frac{x^6}{12}-...)^2}
        \]
        \[
            =\lim_{x\to0}\frac{\frac12-\frac{x^2}{24}+...}{1-2(\frac{x^2}{6}+...)+\frac{(\frac{x^4}{6}+\frac{x^6}{12}-...)^2}{x^2}}
        \]
        Da nun $(\frac{x^4}{6}+\frac{x^6}{12}-...)^2$ einen höheren Grad hat als $x^2$ geht $\frac{(\frac{x^4}{6}+\frac{x^6}{12}-...)^2}{x^2}\to0$ für $x\to0$. \\
        Somit gilt:
        \[
            \lim_{x\to0}\frac{1-cos(x)}{sin^2(x)}
            =\lim_{x\to0}\frac{\frac12}{1}
            =\frac{\frac12}{1}=\frac12
        \]
    \end{solution}
\end{minipage} %
\begin{minipage}[t]{0.5\textwidth}
    \begin{subtask}
        $\lim_{x\to0^+}\frac{1}{3+2^{\frac{1}{x}}}$
    \end{subtask}
    \begin{solution}
        \[
            \lim_{x\to0^+}\frac{1}{3+2^{\frac{1}{x}}}
            =\frac{\lim_{x\to0^+}1}{\lim_{x\to0^+}3+2^{\frac{1}{x}}}
        \]
        \[
            =\frac{1}{3+\lim_{x\to0^+}2^{\frac{1}{x}}}=0
        \]
    \end{solution}
\end{minipage}\\

\begin{minipage}[t]{0.5\textwidth}
    \begin{subtask}
        $\lim_{x\to0^-}\frac{1}{3+2^{\frac{1}{x}}}$
    \end{subtask}
    \begin{solution}
        \[
            \lim_{x\to0^-}\frac{1}{3+2^{\frac{1}{x}}}
            =\frac{\lim_{x\to0^-}1}{\lim_{x\to0^-}3+2^{\frac{1}{x}}}
        \]
        \[
            =\frac{1}{3+\lim_{x\to0^-}2^{\frac{1}{x}}}
            =\frac{1}{3+\lim_{x\to0^-}\frac{1}{2^{\frac{1}{x}}}}=\frac13
        \]
    \end{solution}
\end{minipage} %
\begin{minipage}[t]{0.5\textwidth}
    \begin{subtask}
        $\lim_{x\to0}\frac{1}{3+2^{\frac{1}{x}}}$
    \end{subtask}
    \begin{solution}
        Es gibt keinen eindeutigen Grenzwert hier, da in b) und c) gezeigt wird, dass der rechtsseitige und linksseitige Grenzwert dieses Terms verschieden sind.
    \end{solution}
\end{minipage}