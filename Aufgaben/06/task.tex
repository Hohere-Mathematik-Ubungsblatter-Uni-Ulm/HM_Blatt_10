\task{Existenz von Grenzwerten}{4}
Gegeben seien zwei stetige Funktionen $f:\RR\to\RR$ und $g:\RR\to\RR$. Zudem existiere $\lim_{h\to0}\frac{f(x+h)-f(x)}{h}$ sowie $\lim_{h\to0}\frac{g(x+h)-g(c)}{h}$. Überprüfe, ob dann auch $\lim_{h\to0}\frac{f(x+h)g(x+h)-f(x)g(x)}{h}$ existiert. Was lässt sich mit dem Ergebnis aussagen?

\begin{solution}
        
    Wir formen zuerst $f(x+h)g(x+h) - f(x)g(x)$ um:    
    \[
    f(x+h)g(x+h)-f(x)g(x) 
    = (f(x+h)-f(x))g(x+h)+f(x)(g(x+h)-g(x))
    \]
    \[
    \Rightarrow \frac{f(x+h)g(x+h)-f(x)g(x)}{h}
    =\frac{(f(x+h)-f(x))g(x+h)}{h}+\frac{f(x)(g(x+h)-g(x))}{h}
    \]
    Betrachten wir zuerst den ersten Summanden:
    \[
    \frac{(f(x+h) - f(x))g(x+h)}{h}.
    \]
    Da $g$ stetig ist, gilt $g(x+h) \to g(x)$ für $h \to 0$. Zudem existiert der Grenzwert $\lim_{h \to 0} \frac{f(x+h) - f(x)}{h}$.    
    Bezeichnen wir diesen Grenzwert mit $f'(x)$    
    \[
    \Rightarrow \lim_{h \to 0} \frac{(f(x+h) - f(x))g(x+h)}{h} = f'(x)g(x)
    \]
    Das gleiche machen wir nun mit dem zweiten Summanden:
    \[
    \frac{f(x)(g(x+h) - g(x))}{h}.
    \]
    Da $f(x)$ unabhängig von $h$ ist, kann es aus dem Bruch herausgezogen werden. Weiterhin existiert der Grenzwert $\lim_{h\to0}\frac{g(x+h)-g(x)}{h}$
    Bezeichnen wir diesen Grenzwert mit $g'(x)$
    \[
    \Rightarrow \lim_{h\to0}\frac{f(x)(g(x+h)-g(x))}{h}
    =f(x)g'(x)
    \]
    Kombinieren wir die Ergebnisse der beiden Summanden, erhalten wir:
    \[
    \lim_{h\to0}\frac{f(x+h)g(x+h)-f(x)g(x)}{h}
    =f'(x)g(x)+f(x)g'(x)
    \]
    $\Rightarrow$ Der Grenzwert existiert und entspricht der Produktregel der Differentiation und zeigt, dass das Produkt zweier differenzierbarer Funktionen wieder differenzierbar ist.
\end{solution}