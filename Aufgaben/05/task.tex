\task{Extremwerte und Nullstellen bestimmen}{5}
Die Abbildung $g:[0.2,2]\to\RR$ sei definiert durch
\[g(x)=ln(x)+\frac{(x-1)^3}{17}+sin(x-1).\]
Zeige, dass die Abbildung $g$ auf ihrem Definitionsbereich eine Nullstelle hat sowie ihr Maximum und Minimum annimmt.

\begin{solution}
    Wir untersuchen, ob $g(x)$ auf dem Definitionsbereich stetig ist. \\
    %Satz von Minimum und Maximum ist das Ziel hier
    \(\ln(x), \frac{(x-1)^3}{17} \text{ und } sin(x-1) \text{ sind auf } [0.2,2] \text{ stetig.}\) \\
    $\Rightarrow g(x)=ln(x)+\frac{(x-1)^3}{17}+sin(x+1) $ ist als komposition der stetigen Funktionen auch stetig. \\
    $\Rightarrow $ somit nimmt $g(x)$ nach dem Satz von Minimum und Maximum ein Maximum und Minimum an. \\[2em]
    Nun untersuchen wir $g(x)$ auf eine Nullstelle im Definitionsbereich: \\
    \[f(0.2)=ln(0.2)+\frac{(0.2-1)^3}{17}+sin(0.2-1)\approx -1.654\]
    und 
    \[f(2)=ln(2)+\frac{(2-1)^3}{17}+sin(2-1)\approx 0,769\]
    Somit ereignet sich in der stetigen Funktion $g(x)$ ein Vorzeichenwechsel\\
    $\Rightarrow$ Die Abbildung $g$ hat mindestens eine Nullstelle
\end{solution}